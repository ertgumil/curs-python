\chapter{Control de fluxe}


Fins ara hem vist variables i operacions, però la part més intensa en cómput i la que determina la variabilitat de la nostra execució són els bucles i les operacions condicionals. Realitzar una acció o un altra, o iterar diverses longituds ens vindrà determinat per condicions lògiques.



\section{Lògica booleana}

La lògica booleana ens permet crear funcions lògiques i condicions que determinaran el camí que recorre el nostre codi. A continuació llistem les taules de la veritat de diversos operadors



\begin{table}[!h]
    \begin{centering}
    \begin{tabular}{cc|c|c|c|c|}
    x & y & $x \wedge y$ & $x \lor y$ & $x \oplus y$ & $\neg y$\\ \hline
    0 & 0 & 0 & 1 & 0 & 1 \\
    0 & 1 & 0 & 1 & 1 & 0 \\
    1 & 0 & 0 & 1 & 1 & \\
    1 & 1 & 1 & 1 & 0 &
    \end{tabular}
    \caption{Taula de la veritat}
    \label{tab:python-term}
    \end{centering}
\end{table}

En la taula anterior el valor

\begin{itemize}
\item $\wedge$: conjunció ({\tt \&}). Tots valors han de ser certs.
\item $x \lor y$: disjunció({\tt |}). Un dels dos valors ha de ser cert.
\item $x \oplus y$: disjunció exclusiva(\^{}). Els elements han de ser diferents.
\item $\neg y$: negació({\tt not}). Es nega el valor.
\end{itemize}

En Python "veritat" es representa amb la praula {\tt True} o el número {\tt 1} i "fals" amb la paraula {\tt False} o el número 0.

\begin{tip}[caption=Operador lògics]
>>> True & 1
True
>>> 0 and True
False
>>> True or False
True
>>> True ^ False
True
>>> not True
False
>>> not(True | False)
False
\end{tip}




\section{Operacions condicionals: if, elif, else}

A continuació veurem les condicions. Python utilitza tabulacions per a mostrar diversos nivells del codi. Els operadors que emprem per a l'execució de condicions són {\tt if, elif, else}. L'estructura de les condicions és la següent.

\begin{blockcode}
if expressio1:
   accions1
elif expressio2:
   accions2
elif expressio3:
   accions3
else:
   resta de casos
\end{blockcode}

La primera condició sempre ha de ser un {\tt if} seguida d'una condició. Després, si volem, podem afegir més condicions amb {\tt elif} i considerar la resta de casos (sense condició) amb {\tt else}.

Per exemple en el cas que vulguem saber si una persona és jove, es adulta o si encara no ha nascut respondrem diferentment depenent de les edats. Així doncs creem diverses condicions i prenem una decisió quan aquesta és certa. 

\begin{tip}[caption=Rangs d'edats]
edat=-2
if edat < 18 and edat >= 0:
   print("Jove")
elif edat >= 18:
   print("Adult")
else:
   print("Encara no ha nascut")
\end{tip}




\section{Bucle for}

El bucle {\tt for} ens permet declarar un bucle. En Python s'utilitza, majoritàriament per recorrer un conjunt d'elements tal i com diccionaris, llistes, tuples o altres elements iterables.

\begin{blockcode}
for i in llista:
    realitza acció sobre element i
\end{blockcode}

L'element {\tt i} (que pot ser anomenat d'altra manera) contindrà cada element de la llista a mesura que realitza les accions declarades dins del bucle. Quan hagi recorregut tots els elements sortirà de la iteració.

\begin{tip}[caption=Bucle for]
for i in [0, 1, 2]:
    print("Hola " + str(i))
print("Final")
\end{tip}


La funció {\tt range()} ens permet crear dinàmicament llistes sobre les quals iterar. Si li passem un paràmetre crearà una llista desde 0 fins al valor passat. Amb dos valors li donem un intérval, i amb tres un interval i un valor d'increment. El passem com a paràmetre a la funció {\tt list()} per a forçar que sigui una llista.

\begin{blockcode}
>>> list(range(3))
[0, 1, 2]
>>> list(range(3,9))
[3, 4, 5, 6, 7, 8]
>>> list(range(3,9,2))
[3, 5, 7]
\end{blockcode}

Podem realitzar el codi anterior iterant directament sobre la llista d'elements creats.

\begin{blockcode}
for i in range(2):
    print("Hola " + str(i))
print("Final")
\end{blockcode}

En el cas que volguem recorrer una matriu que és una llista de llistes podem imbrincar els bucles {\tt for}. Així doncs

\begin{tip}[caption=Bucles imbrincats]
m = [[1,2,3],[4,5,6]]
for i in m:
	for j in i:
		print(j)
\end{tip}



\section{Bucle while}

El bucle {\tt while} és emprat en el cas de condicions d'execució i no tan per iteracions, tal i com el bucle {\tt for}. S'executarà sempre mentre que la condició del bucle sigui veritat. Quan sigui fals, s'aturarà.

\begin{blockcode}
while condicio:
  accio
\end{blockcode}

En el següent exercici calculem els quadrats de nombres sense realitzar multiplicacions fins arribar a 5 i imprimim els valors.

\begin{tip}[caption=Càlcul de quadrats amb el bucle while]
i = 0
c = 0
while i < 5:
	c = c + i + i + 1
	i = i + 1
	print(c)
\end{tip}


\section{Break continue pass}

Hi han un conjunt de funcions que ens permet sortir del bucle, saltar a la següent iteració o introduïr una operació buida.

En el cas de la comanda {\tt continue} volem que el programa segueixi l'execució, però que passem a la següent iteració. La comanda {\tt break} sortirà del bucle en quan s'executi. La comanda {\tt pass} no realitza cap acció. Pot ser utilitzat quan una instrucció és requerida sintàcticament però el programa no requereix cap acció, per exemple:


\begin{tip}[caption=Break continue pass]
acc = 0
for i in range(1,10):
	if i % 2 == 0:
		pass
	elif i % 5 == 0:
		break
	elif i % 3 == 3:
		acc = acc + 1
	else:
		continue
	acc = acc + 1
print(acc)
\end{tip}



\subsubsection*{Exercici \Roman{exercici}} \stepcounter{exercici}


Calcula la mitjana i la moda d'una llist a de valors.




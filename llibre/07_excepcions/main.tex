\chapter{Excepcions}


\section{Gestionar excepcions}

Les excepcions són una manera de tractar casos en els quals Python considera que hi ha hagut una excepció de l'execució o interpretació. Per exemple una divisió entre 0 ens donarà una excepció de tipus {\tt ZeroDivisionError}. Python pararà l'execució del nostre programa i ens avisarà d'on ha succeit i quina mena d'error és.


\begin{blockcode}
>>> 2/0
------------------------------------------------------------
ZeroDivisionError          Traceback (most recent call last)
/home/albert/<ipython-input-2-897f73788bb0> in <module>()
----> 1 2/0

ZeroDivisionError: integer division or modulo by zero
\end{blockcode}

A continuació es presenta una llista dels errors de Python extret de la documentació oficial de Python. En aquesta llista hi ha un ordre jeràrquic i algunes excepcions hereten de prèvies. Per exemple, dintre dels errors de sintaxis trobem l'excepció de sagnat.


\begin{blockcode}
BaseException
 +-- SystemExit
 +-- KeyboardInterrupt
 +-- GeneratorExit
 +-- Exception
      +-- StopIteration
      +-- StandardError
      |    +-- BufferError
      |    +-- ArithmeticError
      |    |    +-- FloatingPointError
      |    |    +-- OverflowError
      |    |    +-- ZeroDivisionError
      |    +-- AssertionError
      |    +-- AttributeError
      |    +-- EnvironmentError
      |    |    +-- IOError
      |    |    +-- OSError
      |    |         +-- WindowsError (Windows)
      |    |         +-- VMSError (VMS)
      |    +-- EOFError
      |    +-- ImportError
      |    +-- LookupError
      |    |    +-- IndexError
      |    |    +-- KeyError
      |    +-- MemoryError
      |    +-- NameError
      |    |    +-- UnboundLocalError
      |    +-- ReferenceError
      |    +-- RuntimeError
      |    |    +-- NotImplementedError
      |    +-- SyntaxError
      |    |    +-- IndentationError
      |    |         +-- TabError
      |    +-- SystemError
      |    +-- TypeError
      |    +-- ValueError
      |         +-- UnicodeError
      |              +-- UnicodeDecodeError
      |              +-- UnicodeEncodeError
      |              +-- UnicodeTranslateError
      +-- Warning
           +-- DeprecationWarning
           +-- PendingDeprecationWarning
           +-- RuntimeWarning
           +-- SyntaxWarning
           +-- UserWarning
           +-- FutureWarning
	   +-- ImportWarning
	   +-- UnicodeWarning
	   +-- BytesWarning
\end{blockcode}  
	   


Per a capturar una excepció utilitzarem les funcionalitats {\tt try \& except}. Per exemple podem dividir tots els nombres enters en el rang $[10,-10]$ entre sí mateixos i esperar a que hi hagi una divisió entre zero. 

\begin{tip}[caption=Divisió entre zero]
for i in range(10,-10,-1):
     try:
         i/i
     except ZeroDivisionError:
         print ("Divisio entre 0")
\end{tip}


També podem considerar diversos tipus d'error:

\begin{tip}[caption=Diversos errors]
for i in range(10,-10,-1):
     try:
         i/i
     except TypeError:
         print ("Type Error")
     except ValueError:
         print ("Value Error")
     except:
         print ("Un altre error")
\end{tip}




\section{Llençar excepcions}

Tot i que en els exemples previs hem mostrat l'error mitjançant la funció {\tt print()}, la manera correcta és llençar una excepció, a no ser que ens interessi que l'excepció no aturi el nostre codi. Mitjançant la comanda {\tt raise} nosaltres podem llençar les nostres pròpies excepcions. Podem especificar el tipus d'error i quin és el missatge que volem que es mostri al usuari del nostre programa.


\begin{tip}[caption=Llençar errors amb raise]
for i in range(10,-10,-1):
     if(i==0):
     	raise ValueError("No s'accepta el valor 0")
     i/i
\end{tip}


\subsubsection*{Exercici \Roman{exercici}} \stepcounter{exercici}

Buscar tots els casos pels que és cert l'últim teorema de fermat per a n < 10 i valors de z < 10000.

$x^n + y^n = z^n$
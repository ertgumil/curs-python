


\chapter{Introducció}
\setcounter{page}{1}
\pagestyle{plain}   % Numeració al peu de la pàgina


\section{Què és Python?}

Python és un llenguatge de programació interpretat creat per Guido Van Rossum l'any 1991, al qual li va possar aquest nom en honor al grup còmic \emph{Monty Python}. El llenguatge és lliure i de codi obert; aquesta filosofia permet que tothom tingui accés al codi font per a contribuir-hi, modificar-lo per a les seves necessitats i distriuir-lo lliurement. Python és un llenguatge de \emph{scripting} com els llenguatges de programació Bash o Perl i està programat en C i C++. Una de les ventatges de Python és la seva facilitat d'aprenentatge mitjançant la seva terminal. Python és indepent de plataforma. Podrem executar el nostre codi tan en Windows, Mac OS, Linux, etc. A més podrem compartir el nostre codi en plataformes on-line tal i com iPython Notebook. En l'actualitat té multitud de llibreries: àlgebra lineal, bioinformàtica, visualització de dades, estadística, etc. S'empra en diferents àrees per la seva facilitat d'aprenentatge respecte a altres llenguatges de més baix nivell com C o Fortran, i per la seva rapidessa per a produir codi i entendre'l. És utilitzat a empreses, organitzacións i projectes tal i com Google, CERN, Amazon, YouTube, Anaconda Server o SAGE. 
  

\section{Estat de Python}

Python es troba en transició de la versió 2 a la 3. Per a dur-la a terme hi han programes per a transformar el nostre codi d'una versió a l'altre com \emph{2to3}. En el curs aprendrem Python 3.


\section{Sobre aquest llibre}


Aquest llibre tracta aspectes bàsics de la programació en Python i no arriba a temes tal i com les classes o la programació orientada a objectes. El seu enfocament és 

coneixements bàsics de programació

Va acompanyat d'exemples

Distribució de capítols

Qualsevol correcció, contribució, etc que es vulgui fer a aquest llibre, si us plau enviar e-mail a albertgumi@gmail.com.

Manances que es poden considerar necessàries

Enllaç al codi font en Latex


Els exemple de codi Python

\begin{blockcode}
>>> Codi en Python
Sortida de l'execució
\end{blockcode}

On els caràcters {\tt >>>} indiquen que s'ha d'introduir la comanda a la terminal de Python.


En gris es trobaran els exemples de codi que es troben al fitxer \emph{curs.py}

El contingut dels exemples serà el mateix, tot i que en el codi trobarem més impresions per pantalla bla bla \emph{Exemple 0} significa que la funció que s'ha de cridar és

\begin{blockcode}
>>> import curs
>>> curs.ex000()
Has aconseguit cridar el fitxer de codi
\end{blockcode}

Dir on s'explica cada concepte


S'utilitzarà la lletra verbatim per definir funcions {\tt funcio()}, que sempre tindrà un parèntesi al final i noms de paquests, variables, atributs, etc: {\tt mòdul}.






\chapter{Introducció}
\setcounter{page}{1}
\pagestyle{plain}   % Numeració al peu de la pàgina


\section{Què és Python?}

Python és un llenguatge de programació interpretat creat per Guido Van Rossum l'any 1991, al qual li va posar aquest nom en honor al grup còmic \emph{Monty Python}. El llenguatge és lliure i de codi obert; aquesta filosofia permet que tothom tingui accés al codi font per a contribuir-hi, modificar-lo per a les seves necessitats i distriuir-lo lliurement. Python és un llenguatge de \emph{scripting} com els llenguatges de programació Bash o Perl i està programat en C i C++. Una de les avantatges de Python és la seva facilitat d'aprenentatge mitjançant la seva terminal. Python és indepent de plataforma. Podrem executar el nostre codi tan en Windows, Mac OS, Linux, etc. A més podrem compartir el nostre codi en plataformes on-line tal i com iPython Notebook. En l'actualitat té multitud de llibreries: àlgebra lineal, bioinformàtica, visualització de dades, estadística, etc. S'empra en diferents àrees per la seva facilitat d'aprenentatge respecte a altres llenguatges de més baix nivell com C o Fortran, i per la seva rapidesa per a produir codi i entendre'l. És utilitzat a empreses, organitzacions i projectes tal i com Google, CERN, Amazon, YouTube, Anaconda Server o SAGE. 
  

\section{Estat de Python}

Python es troba en transició de la versió 2 a la 3. Per a dur-la a terme hi han programes per a transformar el nostre codi d'una versió a l'altre com \emph{2to3}. En el curs aprendrem Python 3.


\section{Sobre aquest llibre}


Aquest llibre tracta aspectes bàsics de la programació en Python i no arriba a temes tal i com les classes o la programació orientada a objectes. Busca apropar coneixements bàsics de programació.

Aquest llibre va acompanyat d'un fitxer amb diversos codi d'exemple del llibre. Es pot descarregar des del repositori principal del llibre. En gris es trobaran aquests exemples de codi que estan al fitxer \emph{curs.py}. El contingut dels exemples serà el mateix, tot i que en el codi trobarem més impresions per pantalla dels resultats. Els exemples estan enumerats. \emph{Exemple 0} significa que la funció que s'ha de cridar és la {\tt ex000()}. Per a que el fitxer funcioni han d'estar instal·lades les llibreries NumPy, SciPy i Matplotlib, si no els {\tt import} donaran error.

\begin{blockcode}
>>> import curs
>>> curs.ex000()
Has aconseguit cridar el fitxer de codi
\end{blockcode}

La resta d'exemples es trobaran en color blau i els caràcters {\tt >>>} indiquen que s'ha d'introduir la comanda a la terminal de Python. 

\begin{blockcode}
>>> Codi en Python
Sortida de l'execució
\end{blockcode}



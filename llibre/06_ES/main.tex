\chapter{Escriptura i lectura de fitxers}

La terminal ens ofereix una interfície volàtil per als nostres càlculs. De vegades estem interessats en guardar fitxers amb els resultats del nostre processament, o bé carregar fitxers per a llegir l'entrada del nostre sistema.

\section{Lectura de fixers}

Per a obrir un fitxer emprarem la funció {\tt open()} i per a tancar la funció {\tt close()}. Al obrir hem d'escollir quina serà la interacció que tindrem amb el fitxer. Tenim diverses modes 

\begin{itemize}
\item {\tt a}: Afegeix contingut al final del fitxer.
\item {\tt r}: Només lectura.
\item {\tt w}: Sobreescriure el fitxer existent amb nou contingut.
\end{itemize}

Al obrir el primer paràmetre serà la ruta de l'arxiu i el segon el mode d'obertura. És important que sempre que s'obri un arxiu es tanqui, sinó el nostre descriptor de fitxer quedarà obert pel sistema operatiu.

\begin{blockcode}
f = open('arxiu.txt', 'r')
f.close()
\end{blockcode}

Un cop hem obert el fitxer podrem llegir. La lectura del fitxer anirà avançant a mesura que nosaltres anem obtenint línies de text. Tenim diverses maneres de llegir el fitxer. Podem utilitzar les funcions: 

\begin{itemize}
\item read(): per a llegir tot el fitxer en una cadena de caràcters.
\item readline(): per a llegir línia per línia. Haurem de cridar-la mentre no arribem al final del fitxer.
\item readlines(): per a llegir tot en una llista.
\end{itemize}

Així doncs pel següent text guardat a \emph{arxiu.txt}

\begin{blockcode}
La primera
la segona
i la tercera
\end{blockcode}

Si utilitzem la funció {\tt read()} tot el contingut serà una cadena de text.

\begin{blockcode}
>>> f = open('arxiu.txt', "r")
>>> text = f.read()
>>> print(text)
La primera
la segona
i la tercera
\end{blockcode}

Amb {\tt readline()} haurem de cridar diverses vegades la funció.

\begin{blockcode}
>>> text = f.readline()
>>> print(text)
La primera
>>> text = f.readline()
>>> print(text)
la segona
>>> text = f.readline()
>>> print(text)
i la tercera
\end{blockcode}  

La funció {\tt readlines()} retornarà el contingut en una llista on cada element és una línia del text.

\begin{blockcode}
>>> text = f.readlines()
>>> print(text)
['La primera\n', 'la segona\n', 'i la tercera']
\end{blockcode}  

\section{Escriptura de fitxers}

L'escriptura és realitza en una forma similar que la lectura. Haurem d'utilitzar el mode {\tt a} o {\tt r}. Només tenim dues funcions en aquest cas.

\begin{itemize}
\item {\tt write()}: Escriu directament tot el text
\item {\tt writelines()}: Escriu un text en format llista.
\end{itemize}

\begin{blockcode}
file = open("newfile.txt", "w")
file.write("hello world in the new file\n")
file.writelines(["linea n\n","n+1\n"])
file.close()
\end{blockcode}


\subsubsection*{Exercici \Roman{exercici}} \stepcounter{exercici}

Utilitzant el fitxer anterior \emph{arxiu.txt} buscar la següent subcadena de caràcters programant una funció: "\emph{a t}".


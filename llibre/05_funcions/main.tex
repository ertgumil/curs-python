\chapter{Funcions}


Una funció realitza un conjunt de tasques i pot tornar o no un valor. Si, per exemple tenim la funció $f(x) = sin(x)$, llavors tenim que el nom de la funció és $f$, el paràmetre és $x$ i la funcionalitat i sortida del sistema és el resultat de l'operació $sin(x)$. Per a declarar una funció comencem amb la paraula clau {\tt def} i seguit pel nom de la funció. Els valors calculats es retornen mitjançant la paraula clau {\tt return}. La sortida del sistema retorna per {\tt return} serà només una variable. Si volem tornar diversos valors haurem d'utilitzar llistes o altres estructures. Per tant la definició de la funció que em vist abans seria, en Python.


\begin{tip}[caption=Definició funció f()]
def f(x):
    return sin(x)
\end{tip}


\section{Funcions}

Una funció realitza un conjunt de tasques i pot o no retornar un valor. En el cas que no retorni cap valor, per defecte tindrem el valor $None$. En el cas dels paràmetres podem definir tants com vulguem o cap. En el cas que no passem cap argument a la funció deixarem els parèntesis en blanc {\tt def funcio()}

\begin{blockcode}
def f():
    print("Dintre")
>>> r = f()
>>> print(r)
\end{blockcode}



Podem declarar variables dins de les funcions que només tindran aquest àmbit


\begin{blockcode}
def funcio():
     var = 10
     print(var)
>>> funcio()
10
>>> var
----------------------------------------------------------
NameError          Traceback (most recent call last)
/home/albert/<ipython-input-88-5d73f6e1684c> in <module>()
----> 1 var

NameError: name 'var' is not defined
\end{blockcode}



La sortida del sistema retorna per {\tt return} serà només una variable. Si volem tornar diversos valors haurem d'utilitzar llistes o altres estructures.


\begin{tip}[caption=Retornar llista de valors]
def fact(x):
    llista = []
    for i in range(1,x+1):
        llista.append(math.factorial(i))
    return llista
>>> l = fact(4)
>>> l
[1, 2, 6, 24]
\end{tip}



Una funció no ha de tindre obligatòriament la funció {\tt return}, tot i que sempre tornarà un valor. Si no li especifiquem cap valor ens retornarà {\tt None}.

\begin{blockcode}
def imprimeix_matriu (matriu):
    for i in range(0,len(matriu)):
        for j in range(0,len(matriu[0])):
            print(matriu[i][j])
>>> r = imprimeix_matriu([[1,2],[3,4]])
1
2
3
4
>>> print(r)
None
\end{blockcode}

Podem tindre diverses instàncies de {\tt return} i depenen del flux del programa retornar un valor o un altre.

\begin{tip}[caption=Diversos return]
def bool_a_binari(logic):
    if logic == True:
        return 1
    else:
        return 0
>>> curs.bool_a_binari(True)
1
\end{tip}


Podem especificar valors per defecte de la funció, i donar valors concrets al cridar-la. Per defecte anirà en ordre la crida, però podem especificar arguments en la crida.

\begin{tip}[caption=Valors per defecte i ordre dels paràmetres]
def potencia (x = 2, y = 3):
    return x**y
>>> potencia(2)
8
>>> potencia(y=5, x = -4)
-1024
\end{tip}


\subsubsection*{Exercici \Roman{exercici}} \stepcounter{exercici}

Calcular els primers 1000 nombres prims emprant el sedàs d'Eratòstenes.


\subsubsection*{Exercici \Roman{exercici}} \stepcounter{exercici}

\begin{blockcode}
def mitjana(llista):
    if llista is not list:
        return None
    else:
        ret = 0
        for i in llista:
            ret = ret + i
        retun ret/len(llista)
\end{blockcode}



\subsubsection*{Exercici \Roman{exercici}} \stepcounter{exercici}


Definir la següent funció:

\[
\sum_{i=0}^{n}x^{i}=1+x+x^{2}+x^{3}+\ldots + x^{n}
\]



\subsubsection{Recursivitat}

En Python podem definir funcions recursives. Una funció recursiva és aquella que es crida a si mateixa. Els beneficis de la recursivitat és l'expressivitat, tot i que computacionalment no sempre ens interessarà.


\begin{tip}[caption=Càlcul recursiu d'un nombre factorial]
def factorial(n):
	if(n <= 0):
		return 1
	else:
		return n*fact(n-1)
\end{tip}


		
		
		

\section{Funcions lambda, map i filter}


\subsection{Lambda}

Les funcions lambda són funcions anònimes que utilitzen un constructor anomenat {\tt lambda}. Tenen un sintaxi pròpia i no les podem cridar ni definir de la mateixa forma que les funcions corrents de Python, sinó que té la seva pròpia sintaxis i semàntica.


\begin{tip}[caption=Funcions anònimes]
f = lambda x: math.sin(x)
print(f(3))
g = lambda x,y: -1 * x**y
print(g(1,1))
\end{tip}





\subsection{Map}


La funció {\tt map()} ens és útil quan volem aplicar la mateixa operació a una llista de valors. Per exemple quan volem arrodonir una llista de números. Li passem com a paràmetre el nom de la funció i la llista de valors. Per això, els paràmetres de la funció seran:


\begin{tip}[caption=Calcular el cub d'una llista]
>>> ll = list(range(1,10))
>>> def cub(x): return x**3
>>> list(map(cub,ll))
[1, 8, 27, 64, 125, 216, 343, 512, 729]
\end{tip}



\subsection{Filter}

La funció {\tt filter()} descarta els valors d'una llista quedant-se només amb aquells valors que la funció retorna. Això ens serà útil quan volem descartar valors que no ens interessen d'una llista. El que determinarà si es retorna o no el valor és si la funció d'avaluació retorna cert o fals.


\begin{tip}[caption=Retornarn només valors negatius]
def fil(x):
     if(x < 0):
         return x    
>>> v = list(range(-10,10))
>>> list(filter(fil,v))
[-10, -9, -8, -7, -6, -5, -4, -3, -2, -1]
\end{tip}

Podem combinar les funcions anònimes i les funcions {\tt filter()} i {\tt lambda()}. Si en comptes de passar com a paràmetre el nom de la funció declarem la funció lambda, llavors tindrem una funció més expressiva.

\begin{tip}[caption=Retornarn només valors negatius]
>>> v = list(range(-10,10))
>>> list(filter(lambda x: x < 0,v))
[-10, -9, -8, -7, -6, -5, -4, -3, -2, -1]
\end{tip}



\subsubsection*{Exercici \Roman{exercici}} \stepcounter{exercici}

Realitzar l'exercici anterior de càlcul de nombres prims emprant la funció {\tt filter()}.


\subsubsection*{Exercici \Roman{exercici}} \stepcounter{exercici}


Definir la funció següent:

\[
\sum_{i=0}^{n}x^{i}=1+x+x^{2}+x^{3}+\ldots + x^{n} \left( \left| x \right| <1 \right)
\]



\subsubsection*{Exercici \Roman{exercici}} \stepcounter{exercici}

Defineix l'algoritme del triangle de Tartaglia i executa'l per a 100 nivells.



\subsubsection*{Exercici \Roman{exercici}} \stepcounter{exercici}

Calcular els factors per a qualsevol nombre donat
